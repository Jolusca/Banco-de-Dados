\documentclass[a4paper,12pt]{article}

% Inclui o preâmbulo de um arquivo separado
\usepackage[utf8]{inputenc}
\usepackage[brazil]{babel}
\usepackage{tabularx}
\usepackage{geometry}
\geometry{margin=2.5cm}
\usepackage{setspace}
\usepackage{enumitem}
\usepackage{booktabs} % Para linhas mais bonitas nas tabelas
\usepackage{array}    % Para controle de colunas
\usepackage{listings} % Para inclusão de código
\usepackage{xcolor}   % Para cores no código
\usepackage{graphicx} % Para inclusão de imagens (opcional)
\usepackage{float}

% Configuração para código SQL
\lstset{
    language=SQL,
    basicstyle=\ttfamily\small,
    breaklines=true,
    frame=single,
    captionpos=b,
    keywordstyle=\color{blue},
    commentstyle=\color{green!50!black},
    stringstyle=\color{red},
    showstringspaces=false,
    numbers=left,
    numberstyle=\tiny\color{gray},
    escapeinside={(*@}{@*)},
    morekeywords={JOIN, INNER, LEFT, RIGHT, FULL, CROSS, NATURAL, ON, WHERE, GROUP, HAVING, AVG, SUM, DISTINCT, BETWEEN, IN, NOT, AND, OR, AS, FROM, SELECT, INSERT, UPDATE, DELETE, CREATE, ALTER, DROP, TABLE, DATABASE, INTO, VALUES, SET, PRIMARY, KEY, FOREIGN, REFERENCES}
}

\setstretch{1.2}

\begin{document}

\begin{center}
    \textbf{Universidade Federal do Ceará – UFC}\\
    \textbf{Centro de Ciências – CC}\\
    \textbf{Departamento de Computação – DC}\\
    \textbf{Fundamentos de Bancos de Dados}
\end{center}
\vspace{0.5cm}

\noindent
\textbf{Exercício: SQL Intermediário}\\
\textbf{Data de entrega:} 14/10/2025

\vspace{0.7cm}

\noindent
\textbf{NOME:} João Lucas Oliveira Mota\\
\textbf{MATRÍCULA:} 509597

\vspace{1cm}
\newpage
\begin{center}
    \Large\textbf{Lista de Exercícios SQL - Intermediário}
\end{center}

\vspace{0.5cm}

\section*{Tabelas}

\subsubsection*{Empregado}
\begin{table}[h]
\centering
\caption{Tabela Empregado}
\begin{tabular}{|l|l|l|l|l|l|l|l|}
\hline
\textbf{ENome}   & \textbf{CPF} & \textbf{Endereço} & \textbf{Nasc}   & \textbf{Sexo} & \textbf{Salário} & \textbf{Chefe} & \textbf{Cdep} \\ \hline
Chiquin          & 1234         & rua 1, 1          & 02/02/62        & M             & 10000,00       & 8765          & 3            \\ \hline
Helenita         & 4321         & rua 2, 2          & 03/03/63        & F             & 12000,00       & 6543          & 2            \\ \hline
Pedrin           & 5678         & rua 3, 3          & 04/04/64        & M             & 9000,00        & 6543          & 2            \\ \hline
Valtin           & 8765         & rua 4, 4          & 05/05/65        & M             & 15000,00       & Null          & 4            \\ \hline
Zulmira          & 3456         & rua 5, 5          & 06/06/66        & F             & 12000,00       & 8765          & 3            \\ \hline
Zefinha          & 6543         & rua 6, 6          & 07/07/67        & F             & 10000,00       & 8765          & 2            \\ \hline
\end{tabular}
\end{table}


\subsubsection*{Departamento}
\begin{table}[h]
\centering
\caption{Tabela Departamento}
\begin{tabular}{|l|l|l|}
\hline
\textbf{DNome}      & \textbf{Código} & \textbf{Gerente} \\ \hline
Pesquisa           & 3               & 1234             \\ \hline
Marketing          & 2               & 6543             \\ \hline
Administração      & 4               & 8765             \\ \hline
\end{tabular}
\end{table}

\subsubsection*{Projeto}
\begin{table}[h]
\centering
\caption{Tabela Projeto}
\begin{tabular}{|l|l|l|l|}
\hline
\textbf{PNome}         & \textbf{PCódigo} & \textbf{Cidade}   & \textbf{Cdep} \\ \hline
ProdutoA              & PA               & Cumbuco          & 3            \\ \hline
ProdutoB              & PB               & Icapuí           & 3            \\ \hline
Informatização        & Inf              & Fortaleza        & 4            \\ \hline
Divulgação            & Div              & Morro Branco     & 2            \\ \hline
\end{tabular}
\end{table}


\subsubsection*{Tarefa}
\begin{table}[H]
\centering
\caption{Tabela Tarefa}
\begin{tabular}{|l|l|l|}
\hline
\textbf{CPF} & \textbf{PCódigo} & \textbf{Horas} \\ \hline
1234         & PA               & 30.0            \\ \hline
1234         & PB               & 10.0            \\ \hline
4321         & PA               & 5.0             \\ \hline
4321         & Div              & 35.0            \\ \hline
5678         & Div              & 40.0            \\ \hline
8765         & Inf              & 32.0            \\ \hline
8765         & Div              & 8.0             \\ \hline
3456         & PA               & 10.0            \\ \hline
3456         & PB               & 25.0            \\ \hline
3456         & Div              & 5.0             \\ \hline
6543         & PB               & 40.0            \\ \hline
\end{tabular}
\end{table}

\subsubsection*{DUnidade}
\begin{table}[H]
\centering
\caption{Tabela DUnidade}
\begin{tabular}{|l|l|}
\hline
\textbf{DCódigo} & \textbf{DCidade} \\ \hline
2                & Morro Branco     \\ \hline
3                & Cumbuco          \\ \hline
3                & Prainha          \\ \hline
3                & Taíba            \\ \hline
3                & Icapuí           \\ \hline
4                & Fortaleza        \\ \hline
\end{tabular}
\end{table}


\vspace{1cm}

\section{ Questão 2}

O código para as respostas dessa Questão está no arquivo anexado Q2.
\vspace{1cm}

\begin{flushright}
\textit{“A leitura do mundo precede a leitura da palavra.”}\\
Paulo Freire
\end{flushright}

\newpage
\section*{Códigos SQL Anexados}
\subsection{Questão 2}
\lstinputlisting[
    caption={Consultas SQL: Questão 1},
    label=lst:consultas,
]{lista2-jl.sql}


\end{document}
