%%%%%%%%%%%%%%%%%%%%%%%%%%%%%%%%%%%%%%%%%%%%%%%%%%%%%%%%%%%%%%%%%%%%%%%%%%%%%%%%%%%
%% This project aims to create the UFC template for presentation.                %%
%% author: Maurício Moreira Neto - Doctoral student in Computer Science (MDCC)   %%
%% contacts:                                                                     %%
%%    e-mail: maumneto@ufc.br                                                    %%
%%    linktree: https://linktr.ee/maumneto                                       %%
%%%%%%%%%%%%%%%%%%%%%%%%%%%%%%%%%%%%%%%%%%%%%%%%%%%%%%%%%%%%%%%%%%%%%%%%%%%%%%%%%%%
\documentclass{libs/ufc_format}
% Inserting the preamble file with the packages
\input{libs/preamble.tex}
% Inserting the references file
\bibliography{references.bib}
%\setbeamertemplate{itemize item}{\textbullet} % Marcador padrão (bolinha)
\setbeamertemplate{itemize subitem}{\large\textbullet} % Marcador para subitens (traço)
% Title
\title[Junções]{\huge\textbf{Junções}}
% Subtitle
\subtitle{Álgebra Relacional}
% Author of the presentation
\author{João Lucas Oliveira Mota}

% Institute's Name
\institute[UFC]{
    % email for contact
    \normalsize{\email{jlucasolivera2002@gmail.com}}
    \newline
    % Department Name
    \department{Departamento de Teleinformática}
    \newline
    % university name
    \ufc
}
% date of the presentation
\date{\today}


%%%%%%%%%%%%%%%%%%%%%%%%%%%%%%%%%%%%%%%%%%%%%%%%%%%%%%%%%%%%%%%%%%%%%%%%%%%%%%%%%%
%% Start Document of the Presentation                                           %%               
%%%%%%%%%%%%%%%%%%%%%%%%%%%%%%%%%%%%%%%%%%%%%%%%%%%%%%%%%%%%%%%%%%%%%%%%%%%%%%%%%%
\begin{document}
% insert the code style
\input{libs/code_style}

%% ---------------------------------------------------------------------------
% First frame (with tile, subtitle, ...)
\begin{frame}
    \maketitle
\end{frame}

%% ---------------------------------------------------------------------------
% Second frame
\begin{frame}{Sumário}
    \begin{multicols}{2}
        \tableofcontents
    \end{multicols}
\end{frame}

%% ---------------------------------------------------------------------------
% --- Slide de Introdução ---
\section{Introdução}
\begin{frame}{Junção na Álgebra Relacional}
    \begin{itemize}
        \item Operação fundamental para combinar informações de múltiplas relações.
        \item Diferente do Produto Cartesiano, utiliza condições para filtrar combinações relevantes.
        \begin {itemize}
            \item Faz-se primeiro um produto cartesiano e depois uma seleção conforme os dados necessários
        \end {itemize}
        \item Baseia-se em atributos relacionados (chaves estrangeiras e primárias).
        \item Essencial para a recuperação de dados complexos em bancos relacionais.
    \end{itemize}
\end{frame}
%% ---------------------------------------------------------------------------
% --- Junções internas  ---
\section{Junções Internas}
\begin{frame}{Junção Theta ($\bowtie_\theta$)}
    \begin{itemize}
        \item \textbf{Definição}: Forma mais geral de junção. Combina tuplas que satisfazem um predicado $\theta$.
        \item \textbf{Predicado $\theta$}: Operadores de comparação ($=, <, >, \leq, \geq, \neq$).
        \item \textbf{Notação}: $R \bowtie_\theta S = \sigma_\theta(R \times S)$.
    \end{itemize}
    \begin{block}{Exemplo}
        \texttt{Empregado} $\bowtie_{\text{ID\_Emp} > \text{ID\_Dept}}$ \texttt{Departamento}
    \end{block}
\end{frame}

% --- Equijoin ---
\begin{frame}{Junção de Igualdade (Equijoin)}
    \begin{itemize}
        \item Caso especial da Junção Theta onde $\theta$ usa apenas o operador de igualdade ($=$).
        \item Mantém todas as colunas de ambas as relações (gera colunas duplicadas).
    \end{itemize}
    \begin{block}{Exemplo}
        \texttt{Empregado} $\bowtie_{\text{E.Cdep} = \text{D.Código}}$ \texttt{Departamento}
    \end{block}
\end{frame}

% --- Junção Natural ---
\begin{frame}{Junção Natural ($\bowtie$)}
    \begin{itemize}
        \item Realizada sobre atributos com o mesmo nome em ambas as relações.
        \item \textbf{Diferencial}: Elimina automaticamente as colunas duplicadas.
        \item Notação: $R \bowtie S$.
    \end{itemize}
    \begin{block}{Exemplo}
        \texttt{Empregado} $\bowtie$ \texttt{Departamento}
    \end{block}
\end{frame}


%% ---------------------------------------------------------------------------
%% Junção externa

\section{Junções Externas}

\begin{frame}{Junção Externa à Esquerda (Left Outer Join)}
    \begin{itemize}
        \item \textbf{Notação}: $R \rtimes S$
        \item \textbf{Funcionamento}: Preserva todas as tuplas da relação à esquerda ($R$).
        \item \textbf{Correspondência}: Se não houver par em $S$, os atributos de $S$ são preenchidos com \textbf{NULL}.
        \item \textbf{Uso}: Quando a tabela da esquerda é a entidade principal da consulta.
    \end{itemize}
\end{frame}

\begin{frame}{Junção Externa à Direita (Right Outer Join)}
    \begin{itemize}
        \item \textbf{Notação}: $R \ltimes S$
        \item \textbf{Funcionamento}: Preserva todas as tuplas da relação à direita ($S$).
        \item \textbf{Correspondência}: Se não houver par em $R$, os atributos de $R$ são preenchidos com \textbf{NULL}.
        \item \textbf{Uso}: Inverso do Left Join; garante a integridade dos dados da tabela à direita.
    \end{itemize}
\end{frame}

\begin{frame}{Junção Externa Completa (Full Outer Join)}
    \begin{itemize}
        \item \textbf{Notação}: $R \Join S$ % Símbolo padrão de junção
        \item \textbf{Funcionamento}: Preserva todas as tuplas de ambas as relações ($R$ e $S$).
        \item \textbf{Correspondência}: Preenche com \textbf{NULL} em qualquer lado que falte a correspondência.
        \item \textbf{Uso}: Para obter uma visão total de dados relacionados, sem perder registros de nenhum lado.
    \end{itemize}
\end{frame}


\section{Junções Avançadas}
\begin{frame}{Semi-Junção e Anti-Junção}
    \begin{block}{Semi-Junção ($\ltimes$)}
        Retorna tuplas de $R$ que possuem correspondência em $S$, mas sem os atributos de $S$.
        \[ R \ltimes_\theta S = \Pi_{\text{Atributos}(R)}(R \bowtie_\theta S) \]
    \end{block}
    \begin{block}{Anti-Junção ($\triangleright$)}
        Retorna tuplas de $R$ que \textbf{não} possuem correspondência em $S$.
        \[ R \triangleright S \]
    \end{block}
\end{frame}

%% ---------------------------------------------------------------------------
\section{Resumo Geral}

\begin{frame}{Resumo das Operações de Junção}
    \vspace{0.2cm}
    \footnotesize
    \centering
    \renewcommand{\arraystretch}{2} % Aumenta o espaçamento entre as linhas para ficar menos "apertado"
    \setlength{\tabcolsep}{10pt}      % Aumenta o espaço entre as colunas
    
    \begin{tabularx}{\textwidth}{l c l} % O 'X' diz: "use todo o espaço que sobrar"
    \textbf{Operação} & \textbf{Notação} & \textbf{Aplicação Prática} \\
    \hline
        \textbf{Theta / Equi} & $R \bowtie_{\theta} S$ & Combinação por critérios lógicos ou igualdade. \\
        
        \textbf{Natural} & $R \bowtie S$ & Junção automática por nomes iguais e únicos. \\
        
        \textbf{Left Join} & $R \rtimes S$ & Preserva todos os dados da tabela à  \textbf{esquerda}. \\
        
        \textbf{Right Join} & $R \ltimes S$ & Preserva todos os dados da tabela à \textbf{direita}. \\
        
        \textbf{Full Join} & $R \Join S$ & Preserva todos os dados de \textbf{ambos os lados}. \\
        
        \textbf{Semi-Join} & $R \ltimes_{\theta} S$ & Filtra registros que \textbf{possuem} correspondência. \\
        
        \textbf{Anti-Join} & $R \triangleright S$ & Filtra registros que \textbf{não possuem} correspondência. \\
        \hline
    \end{tabularx}

    \vspace{0.5cm}
    \begin{flushleft}
        \scriptsize
        \textbf{Dica:} Use Junções \textbf{Internas} para dados exatos e \textbf{Externas} para evitar perda de registros órfãos.
    \end{flushleft}
\end{frame}


\end{document}