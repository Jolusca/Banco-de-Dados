\documentclass[a4paper,12pt]{article}

% Inclui o preâmbulo de um arquivo separado
\usepackage[utf8]{inputenc}
\usepackage[brazil]{babel}
\usepackage{tabularx}
\usepackage{geometry}
\geometry{margin=2.5cm}
\usepackage{setspace}
\usepackage{enumitem}
\usepackage{booktabs} % Para linhas mais bonitas nas tabelas
\usepackage{array}    % Para controle de colunas
\usepackage{listings} % Para inclusão de código
\usepackage{xcolor}   % Para cores no código
\usepackage{graphicx} % Para inclusão de imagens (opcional)
\usepackage{float}

% Configuração para código SQL
\lstset{
    language=SQL,
    basicstyle=\ttfamily\small,
    breaklines=true,
    frame=single,
    captionpos=b,
    keywordstyle=\color{blue},
    commentstyle=\color{green!50!black},
    stringstyle=\color{red},
    showstringspaces=false,
    numbers=left,
    numberstyle=\tiny\color{gray},
    escapeinside={(*@}{@*)},
    morekeywords={JOIN, INNER, LEFT, RIGHT, FULL, CROSS, NATURAL, ON, WHERE, GROUP, HAVING, AVG, SUM, DISTINCT, BETWEEN, IN, NOT, AND, OR, AS, FROM, SELECT, INSERT, UPDATE, DELETE, CREATE, ALTER, DROP, TABLE, DATABASE, INTO, VALUES, SET, PRIMARY, KEY, FOREIGN, REFERENCES}
}

\setstretch{1.2}

\begin{document}

\begin{center}
    \textbf{Universidade Federal do Ceará – UFC}\\
    \textbf{Centro de Ciências – CC}\\
    \textbf{Departamento de Computação – DC}\\
    \textbf{Fundamentos de Bancos de Dados}
\end{center}
\vspace{0.5cm}

\noindent
\textbf{Exercício: SQL Introdutório}\\
\textbf{Objetivo:} Introduzir os principais conceitos da linguagem SQL.\\
\textbf{Data de entrega:} 14/10/2025

\vspace{0.7cm}

\noindent
\textbf{NOME:} João Lucas Oliveira Mota\\
\textbf{MATRÍCULA:} 509597

\vspace{1cm}

\begin{center}
    \Large\textbf{Lista de Exercícios SQL - Introdutório}
\end{center}

\vspace{0.5cm}

\section{ Questão}
Utilizando o PostgreSQL, crie as relações (tabelas) mostradas a seguir. Em seguida, insira as tuplas (linhas) ilustradas nas figuras. Siga o seguinte roteiro:
\subsection*{a) Identificar o domínio dos atributos, o grau das relações e a diferença entre esquema }
\subsubsection*{Domínio dos atributos}

\begin{itemize}
    \item \textbf{Empregado}:
    \begin{itemize}
        \item \texttt{ENome}: Texto (\texttt{VARCHAR})
        \item \texttt{CPF}: Numérico (\texttt{INTEGER} ou \texttt{VARCHAR})
        \item \texttt{Endereço}: Texto (\texttt{VARCHAR})
        \item \texttt{Nasc}: Data (\texttt{DATE})
        \item \texttt{Sexo}: Texto (\texttt{CHAR(1)})
        \item \texttt{Salário}: Numérico (\texttt{DECIMAL} ou \texttt{FLOAT})
        \item \texttt{Chefe}: Numérico (\texttt{INTEGER}, referenciando \texttt{CPF})
        \item \texttt{Cdep}: Numérico (\texttt{INTEGER}, referenciando \texttt{Código} do Departamento)
    \end{itemize}

    \item \textbf{Departamento}:
    \begin{itemize}
        \item \texttt{DNome}: Texto (\texttt{VARCHAR})
        \item \texttt{Código}: Numérico (\texttt{INTEGER})
        \item \texttt{Gerente}: Numérico (\texttt{INTEGER}, referenciando \texttt{CPF} do Empregado)
    \end{itemize}

    \item \textbf{Projeto}:
    \begin{itemize}
        \item \texttt{PNome}: Texto (\texttt{VARCHAR})
        \item \texttt{PCódigo}: Texto (\texttt{VARCHAR})
        \item \texttt{Cidade}: Texto (\texttt{VARCHAR})
        \item \texttt{Cdep}: Numérico (\texttt{INTEGER}, referenciando \texttt{Código} do Departamento)
    \end{itemize}

    \item \textbf{Tarefa}:
    \begin{itemize}
        \item \texttt{CPF}: Numérico (\texttt{INTEGER}, referenciando \texttt{CPF} do Empregado)
        \item \texttt{PCódigo}: Texto (\texttt{VARCHAR}, referenciando \texttt{PCódigo} do Projeto)
        \item \texttt{Horas}: Numérico (\texttt{DECIMAL} ou \texttt{FLOAT})
    \end{itemize}

    \item \textbf{DUnidade}:
    \begin{itemize}
        \item \texttt{DCódigo}: Numérico (\texttt{INTEGER}, referenciando \texttt{Código} do Departamento)
        \item \texttt{DCidade}: Texto (\texttt{VARCHAR})
    \end{itemize}
\end{itemize}

\vspace{0.5cm}
\subsubsection*{Grau das Relações}
\begin{itemize}
    \item \textbf{Empregado}: 8 atributos
    \item \textbf{Departamento}: 3 atributos
    \item \textbf{Projeto}: 4 atributos
    \item \textbf{Tarefa}: 3 atributos
    \item \textbf{DUnidade}: 2 atributos
\end{itemize}

\subsubsection*{Esquema de Relação vs. Relação}
\begin{itemize}
    \item \textbf{Esquema de relação}: Define a estrutura da tabela (nomes dos atributos e seus domínios).\\
    Exemplo: \texttt{Empregado(ENome, CPF, Endereço, Nasc, Sexo, Salário, Chefe, Cdep)}.
    \item \textbf{Relação}: Conjunto de tuplas (linhas) que obedecem ao esquema. \\
    Exemplo: A tabela \texttt{Empregado} com os dados inseridos.
\end{itemize}

\subsection*{b) Identificar as chaves candidatas, a chave primária e as chaves estrangeiras para cada relação.}

\subsubsection*{Empregado}
\begin{itemize}
    \item \textbf{Chaves candidatas}: \texttt{CPF} (único para cada empregado).
    \item \textbf{Chave primária}: \texttt{CPF}.
    \item \textbf{Chaves estrangeiras}: \texttt{Chefe} (referencia \texttt{CPF} em \texttt{Empregado}), \texttt{Cdep} (referencia \texttt{Código} em \texttt{Departamento}).
\end{itemize}

\subsubsection*{Departamento}
\begin{itemize}
    \item \textbf{Chaves candidatas}: \texttt{Código} (único para cada departamento).
    \item \textbf{Chave primária}: \texttt{Código}.
    \item \textbf{Chaves estrangeiras}: \texttt{Gerente} (referencia \texttt{CPF} em \texttt{Empregado}).
\end{itemize}

\subsubsection*{Projeto}
\begin{itemize}
    \item \textbf{Chaves candidatas}: \texttt{PCódigo} (único para cada projeto).
    \item \textbf{Chave primária}: \texttt{PCódigo}.
    \item \textbf{Chaves estrangeiras}: \texttt{Cdep} (referencia \texttt{Código} em \texttt{Departamento}).
\end{itemize}

\subsubsection*{Tarefa}
\begin{itemize}
    \item \textbf{Chaves candidatas}: \texttt{(CPF, PCódigo)} (combinados, únicos para cada tarefa).
    \item \textbf{Chave primária}: \texttt{(CPF, PCódigo)}.
    \item \textbf{Chaves estrangeiras}: \texttt{CPF} (referencia \texttt{CPF} em \texttt{Empregado}), \texttt{PCódigo} (referencia \texttt{PCódigo} em \texttt{Projeto}).
\end{itemize}

\subsubsection*{DUnidade}
\begin{itemize}
    \item \textbf{Chaves candidatas}: \texttt{(DCódigo, DCidade)} (combinados, únicos para cada unidade).
    \item \textbf{Chave primária}: \texttt{(DCódigo, DCidade)}.
    \item \textbf{Chaves estrangeiras}: \texttt{DCódigo} (referencia \texttt{Código} em \texttt{Departamento}).
\end{itemize}

\subsection*{c) Criar um \textit{Database} para o esquema acima (usar a interface do PostgreSQL).}
Os comandos para criar o banco de dados estão presentes no código Q1.

\subsection*{d) Criar as relações mostradas no esquema acima.}
Os comandos para criar as tabelas estão presentes no código Q1.

\subsection*{e) Povoar as relações.}
Os comandos para inserir os dados nas tabelas estão presentes no código Q1.

\subsubsection*{Empregado}
\begin{table}[h]
\centering
\caption{Tabela Empregado}
\begin{tabular}{|l|l|l|l|l|l|l|l|}
\hline
\textbf{ENome}   & \textbf{CPF} & \textbf{Endereço} & \textbf{Nasc}   & \textbf{Sexo} & \textbf{Salário} & \textbf{Chefe} & \textbf{Cdep} \\ \hline
Chiquin          & 1234         & rua 1, 1          & 02/02/62        & M             & 10000,00       & 8765          & 3            \\ \hline
Helenita         & 4321         & rua 2, 2          & 03/03/63        & F             & 12000,00       & 6543          & 2            \\ \hline
Pedrin           & 5678         & rua 3, 3          & 04/04/64        & M             & 9000,00        & 6543          & 2            \\ \hline
Valtin           & 8765         & rua 4, 4          & 05/05/65        & M             & 15000,00       & Null          & 4            \\ \hline
Zulmira          & 3456         & rua 5, 5          & 06/06/66        & F             & 12000,00       & 8765          & 3            \\ \hline
Zefinha          & 6543         & rua 6, 6          & 07/07/67        & F             & 10000,00       & 8765          & 2            \\ \hline
\end{tabular}
\end{table}


\subsubsection*{Departamento}
\begin{table}[h]
\centering
\caption{Tabela Departamento}
\begin{tabular}{|l|l|l|}
\hline
\textbf{DNome}      & \textbf{Código} & \textbf{Gerente} \\ \hline
Pesquisa           & 3               & 1234             \\ \hline
Marketing          & 2               & 6543             \\ \hline
Administração      & 4               & 8765             \\ \hline
\end{tabular}
\end{table}

\subsubsection*{Projeto}
\begin{table}[h]
\centering
\caption{Tabela Projeto}
\begin{tabular}{|l|l|l|l|}
\hline
\textbf{PNome}         & \textbf{PCódigo} & \textbf{Cidade}   & \textbf{Cdep} \\ \hline
ProdutoA              & PA               & Cumbuco          & 3            \\ \hline
ProdutoB              & PB               & Icapuí           & 3            \\ \hline
Informatização        & Inf              & Fortaleza        & 4            \\ \hline
Divulgação            & Div              & Morro Branco     & 2            \\ \hline
\end{tabular}
\end{table}


\subsubsection*{Tarefa}
\begin{table}[h]
\centering
\caption{Tabela Tarefa}
\begin{tabular}{|l|l|l|}
\hline
\textbf{CPF} & \textbf{PCódigo} & \textbf{Horas} \\ \hline
1234         & PA               & 30.0            \\ \hline
1234         & PB               & 10.0            \\ \hline
4321         & PA               & 5.0             \\ \hline
4321         & Div              & 35.0            \\ \hline
5678         & Div              & 40.0            \\ \hline
8765         & Inf              & 32.0            \\ \hline
8765         & Div              & 8.0             \\ \hline
3456         & PA               & 10.0            \\ \hline
3456         & PB               & 25.0            \\ \hline
3456         & Div              & 5.0             \\ \hline
6543         & PB               & 40.0            \\ \hline
\end{tabular}
\end{table}

\subsubsection*{DUnidade}
\begin{table}[h]
\centering
\caption{Tabela DUnidade}
\begin{tabular}{|l|l|}
\hline
\textbf{DCódigo} & \textbf{DCidade} \\ \hline
2                & Morro Branco     \\ \hline
3                & Cumbuco          \\ \hline
3                & Prainha          \\ \hline
3                & Taíba            \\ \hline
3                & Icapuí           \\ \hline
4                & Fortaleza        \\ \hline
\end{tabular}
\end{table}


\vspace{1cm}

\section{ Questão}
Utilize a linguagem de consulta SQL e recupere as informações solicitadas a seguir:
\begin{enumerate}[label=\alph*)]
    \item Recupere o nome e o salário de todos os empregados que trabalham em Marketing.
    \item Recupere o CPF de todos os empregados que trabalham em Pesquisa ou que diretamente gerenciam um empregado que trabalha em Pesquisa.
    \item Recupere o nome e a cidade dos projetos que envolvem (contêm) pelo menos um empregado que trabalha mais de 30 horas nesse projeto.
    \item Recupere o nome e a data de nascimento dos gerentes de cada departamento.
    \item Recupere o nome e o endereço de todos os empregados que trabalham para o departamento “Pesquisa”.
    \item Para cada projeto localizado em Icapuí, recupere o código do projeto, o nome do departamento que o controla e o nome do seu gerente.
    \item Recupere o nome e o sexo dos empregados que não são gerentes.
\end{enumerate}
\subsubsection*{Resposta:}
O código para as respostas dessa Questão está no arquivo anexado Q2.
\vspace{1cm}

\begin{flushright}
\textit{“A leitura do mundo precede a leitura da palavra.”}\\
Paulo Freire
\end{flushright}

\newpage
\section{Códigos SQL Anexados}
\subsection{Questão 1}
\lstinputlisting[
    caption={Consultas SQL: Questão 1},
    label=lst:consultas,
]{q1.sql}
\newpage
\subsection{Questão 2}
\lstinputlisting[
    caption={Consultas SQL: Questão 2},
    label=lst:consultas,
]{q2.sql}

\end{document}